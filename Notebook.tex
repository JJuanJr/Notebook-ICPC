\documentclass[10pt,landscape,twocolumn,letterpaper,twosided]{article}

\usepackage[spanish]{babel}
\usepackage[utf8]{inputenc}
\usepackage{geometry}
\usepackage{float}
\usepackage{amsmath}
\usepackage{supertabular}
\usepackage{booktabs}
\usepackage{multirow}

\geometry{verbose,landscape,letterpaper,tmargin=2cm,bmargin=2cm,lmargin=1.5cm,rmargin=1.5cm}

\begin{document}
	
	\title{Repositorio en C++}
	\author{Universidad de la Amazonia, Colombia.}
	\maketitle
	
	\tableofcontents
	\hfill
	
	\section{Formulas}

		\subsection{Formulas generales}
			\begin{tabular}{|p{6.5cm}|p{3.9cm}|}
				\hline
				\multicolumn{2}{|c|}{PERMUTACIÓN Y COMBINACIÓN} \\ \hline
				Combinación (Coeficiente Binomial): Número de subconjuntos de k elementos escogidos de un conjunto con n
					elementos.
					& $ \binom{n}{k} = \binom{n}{n-k} = \displaystyle\frac{n!}{k!(n-k)!} $ \\ \hline

				Combinación con repetición: Número de grupos formados por n elementos, partiendo de m tipos de elementos.
					& $ \binom{m+n-1}{n} = \displaystyle\frac{(m + n - 1)!}{n!(m-1)!} $ \\ \hline
		
				Permutación: Número de formas de agrupar n elementos, donde importa el orden y sin repetir elementos
					& $ P_{n} = n! $ \\ \hline

				Permutación múltiple: Elegir r elementos de n posibles con repetición 
					& $ n^{r} $	\\ \hline
	
				Permutación con repetición: Se tienen n elementos donde el primero se repite a veces, el segundo b veces, etc.
					& $ PR_{n}^{a,b,c...} = \displaystyle\frac{P_{n}}{a!b!c!...}$ \\ \hline
	
				Permutaciones sin repetición: Número de formas de agrupar r elementos de n disponibles, sin repetir elementos
					& $\displaystyle\frac{n!}{(n-r)!}$ \\ \hline
			\end{tabular}
			
			\begin{tabular}{|p{2.0cm}|p{2.8cm}|p{2.0cm}|p{2.8cm}|}
				\hline
				\multicolumn{4}{|c|}{CIRCUNFERENCIA Y CÍRCULO} \\  \hline
				\multicolumn{4}{|p{10cm}|}{Considerando $r$ como el radio, $\alpha$ como el ángulo del arco o sector, y (R, r) 
					como radio mayor y menor respectivamente.} \\ \hline
						Área                   & $A = \pi * r^{2} $ & Longitud & $L = 2*\pi*r$  \\ \hline
					Longitud de un arco    & $L = \displaystyle\frac{2*\pi*r*\alpha}{360}$ & Área sector circular 
						& $A = \displaystyle\frac{\pi * r^{2} * \alpha}{360}$ \\ \hline
					Área corona circular   & $A = \pi  (R^{2} - r^{2})$ 
						& Formula general & $(X-P_{x})^{2}+(Y-P_{y})^2=r^{2}$ \\ \hline 
			\end{tabular}
			
			\begin{tabular}{|p{2.0cm}|p{2.8cm}|p{2.0cm}|p{2.8cm}|}
				\hline
				\multicolumn{4}{|c|}{TRIÁNGULO} \\  \hline

				\multicolumn{4}{|p{10cm}|}{Considerando $b$ como la longitud de la base, $h$ como la altura, letras minúsculas 
				como la longitud de los lados, letras mayúsculas como los ángulos, y $r$ como el radio de círcunferencias 
				asociadas.} \\ \hline
	
				Área con base y altura & $A = \displaystyle\frac{1}{2}b * h$ &
					Área con 2 lados y su ángulo & $A = \displaystyle\frac{1}{2}b*a*sin(C)$ \\ \hline
				Área con los 3 lados & \multicolumn{3}{ |c|} {$ A = \sqrt{p(p - a)(p - b)(p - c)}$ con 
					$p = \displaystyle\frac{a + b + c}{2}$ } \\ \hline
				Triángulo circunscrito a circunferencia & $A = \displaystyle\frac{abc}{4r}$ &
					Triángulo inscrito a circunferencia & $A = r(\displaystyle\frac{a+b+c}{2})$ \\ \hline
				Triangulo equilátero & $A = \displaystyle\frac{\sqrt{3}}{4}a^{2}$ & & \\ \hline 
			\end{tabular}

			\begin{tabular}{|p{3.3cm}|p{3.3cm}|p{3.3cm}|}
				\hline
				\multicolumn{3}{|c|}{TRIGONOMÉTRIA} \\ \hline
	
				$sin(\alpha) = \displaystyle\frac{opuesto}{hipotenusa}$ 
					& $cos(\alpha) = \displaystyle\frac{adyacente}{hipotenusa}$
					& $tan(\alpha) = \displaystyle\frac{opuesto}{adyacente}$ \\ \hline
		
				$sec(\alpha) = \displaystyle\frac{1}{cos(\alpha)}$ 
					& $csc(\alpha) = \displaystyle\frac{1}{sin(\alpha)}$
					& $cot(\alpha) = \displaystyle\frac{1}{tan(\alpha)}$ \\ \hline
	
				Ley de los senos, con $\gamma$ el angulo opuesto al lado $c$ 
					& \multicolumn{2}{ |c|} {$\displaystyle\frac{a}{sin(\alpha)} 
						= \frac{b}{sin(\beta)} = \frac{c}{sin(\gamma)}$ } \\ \hline
				Ley de los cosenos, con $\gamma$ el angulo opuesto al lado $c$ 
					& \multicolumn{2}{ |c|} {$ c^{2} = a^{2}+b^{2}-2ab*cos(\gamma)$ } \\ \hline
			\end{tabular}
			
			\begin{tabular}{|p{2.2cm}|p{8.2cm}|}
				\hline
				\multicolumn{2}{|c|}{PROPIEDADES DEL MÓDULO (RESIDUO)} \\  \hline
				Neutra & (a \% b) \% b = a \% b \\ \hline
				Asociativa en suma & (a + b) \% c = ((a \% c) + (b \% c)) \% c \\ \hline
				Asociativa en resta & (a - b) \% c = ((a \% c) - (b \% c) + c) \% c \\ \hline
				Asociativa en multiplicación &  (a*b) \% c = ((a \% c)*(b \% c)) \% c \\ \hline
			\end{tabular}
			
			\begin{tabular}{|p{2.0cm}|p{2.8cm}|p{2.0cm}|p{2.8cm}|}
				\hline
				\multicolumn{4}{|c|}{FIGURAS} \\ \hline
				Elipse & $A = PI*a*b $ & Cono & $ V = \frac{1}{3} * PI*r^{2}*h$ \\ \hline
				Cilindro & $V = PI*r^{2}*h$ & Esfera & $V = \frac{4}{3}*PI*r^{3}$  \\ \hline
			\end{tabular}
			
		\subsection{Tabla ASCII}
			Caracteres ASCII con sus respectivos valores numéricos.
			\begin{table}[H]
				\begin{tabular}{|l|l|l|l|l|l|l|l|}
					\hline \textbf{No.} & \textbf{ASCII} & \textbf{No.} & \textbf{ASCII}  &
						\textbf{No.} & \textbf{ASCII} & \textbf{No.} & \textbf{ASCII} \\ \hline
					32 & space & 40 & ( & 48 & 0 & 56 & 8 \\ \hline
					33 &  ! & 41 & ) & 49 & 1 & 57 & 9 \\ \hline
					34 &  " & 42 & * & 50 & 2 & 58 & : \\ \hline
					35 & \# & 43 & + & 51 & 3 & 59 & ; \\ \hline
					36 & \$ & 44 & , & 52 & 4 & 60 & < \\ \hline
					37 & \% & 45 & - & 53 & 5 & 61 & = \\ \hline
					38 & \& & 46 & . & 54 & 6 & 62 & > \\ \hline
					39 &  ' & 47 & / & 55 & 7 & 63 & ? \\ \hline
				\end{tabular}
			\end{table}
		
			\begin{tabular}{|l|l|l|l|l|l|l|l|}
				\hline \textbf{No.} & \textbf{ASCII} & \textbf{No.} & \textbf{ASCII}  &
					\textbf{No.} & \textbf{ASCII} & \textbf{No.} & \textbf{ASCII} \\ \hline
				64 & @ & 72 & H & 80 & P & 88 & X \\ \hline
				65 & A & 73 & I & 81 & Q & 89 & Y \\ \hline
				66 & B & 74 & J & 82 & R & 90 & Z \\ \hline
				67 & C & 75 & K & 83 & S & 91 & [ \\ \hline
				68 & D & 76 & L & 84 & T & 92 & \textbackslash \\ \hline
				69 & E & 77 & M & 85 & U & 93 & ] \\ \hline
				70 & F & 78 & N & 86 & V & 94 & \textasciicircum \\ \hline
				71 & G & 79 & O & 87 & W & 95 & \_ \\ \hline
			\end{tabular}
		
			\begin{tabular}{|l|l|l|l|l|l|l|l|}
				\hline \textbf{No.} & \textbf{ASCII} & \textbf{No.} & \textbf{ASCII}  &
					\textbf{No.} & \textbf{ASCII} & \textbf{No.} & \textbf{ASCII} \\ \hline
				96 & ` & 104 & h & 112 & p & 120 & x \\ \hline
				97 & a & 105 & i & 113 & q & 121 & y \\ \hline
				98 & b & 106 & j & 114 & r & 122 & z \\ \hline
				99 & c & 107 & k & 115 & s & 123 & \{ \\ \hline
				100 & d & 108 & l & 116 & t & 124 & \textbar \\ \hline
				101 & e & 109 & m & 117 & u & 125 & \} \\ \hline
				102 & f & 110 & n & 118 & v & 126 & \textasciitilde \\ \hline
				103 & g &  111 & o & 119 & w & 127 &  \\ \hline
			\end{tabular}
			
			\subsection{Sequences}
				Listado de secuencias mas comunes y como hallarlas.

				\begin{center}
					\tablefirsthead{}
					\tabletail{
						\midrule 
						\multicolumn{2}{r}{{Continúa en la siguiente columna}} \\}
					\tablelasttail{}
					{
					\renewcommand{\arraystretch}{1.4}
					\begin{supertabular}{|p{1.8cm}|p{8.6cm}|}

						\hline

						\multirow{2}{2cm}{Estrellas octangulares}
						& 	0, 1, 14, 51, 124, 245, 426, 679, 1016, 1449, 1990, 2651, ...
						\\ \cline{2-2}
						& $f(n) = n*(2*n^{2} - 1)$.
						\\ \hline

						\multirow{2}{2cm}
						{Euler totient}    
						& 1, 1, 2, 2, 4, 2, 6, 4, 6, 4, 10, 4, 12, 6,...            
						\\ \cline{2-2} 
						& $f(n) = $ Cantidad de números $\leq n$ coprimos con n. 
						\\ \hline

						\multirow{2}{2cm}
						{Números de Catalán} 
						& 1, 1, 2, 5, 14, 42, 132, 429, 1430, 4862, 16796, 58786, ...
						\\ \cline{2-2}
						& $f(n)=\displaystyle\frac{(2n)!}{(n + 1)! n!}$
						\\ \hline

						\multirow{2}{2cm}{Números de Fermat}
						& 3, 5, 17, 257, 65537, 4294967297, 18446744073709551617, ...
						\\ \cline{2-2}
						& $f(n) = 2^{(\displaystyle2^{\textstyle n})} + 1$
						\\ \hline

						\multirow{2}{2cm}{Números de Pell} 
						& 0, 1, 2, 5, 12, 29, 70, 169, 408, 985, 2378, 5741, 13860, ...
						\\ \cline{2-2} 
						& $f(0) = 0; f(1) = 1; f(n) = 2f(n-1) + f(n-2)$ para $n>1$
						\\ \hline

						\multirow{2}{2cm}{Números piramidales cuadrados}
						& 0, 1, 5, 14, 30, 55, 91, 140, 204, 285, 385, 506, 650, ...
						\\ \cline{2-2}
						& $f(n) = \displaystyle\frac{n*(n+1)*(2*n+1)}{6}$
						\\ \hline

						\multirow{2}{2cm}{Números primos de Mersenne}
						& 3, 7, 31, 127, 8191, 131071, 524287, 2147483647, ...
						\\ \cline{2-2}
						& $f(n) = 2^{p(n)} - 1$ donde $p$ representa valores primos iniciando en $p(0)=2$.
						\\ \hline

						\multirow{2}{2cm}{Números tetraedrales}
						& 0, 1, 4, 10, 20, 35, 56, 84, 120, 165, 220, 286, 364, ...
						\\ \cline{2-2}
						& $f(n) = \displaystyle\frac{n*(n+1)*(n+2)}{6}$
						\\ \hline

						\multirow{2}{2cm}{Números triangulares}
						& 0, 1, 3, 6, 10, 15, 21, 28, 36, 45, 55, 66, 78, 91, 105, ...
						\\ \cline{2-2}
						& $f(n) = \displaystyle\frac{n(n+1)}{2}$
						\\ \hline

						\multirow{2}{2cm}{OEIS A000127}
						& 1, 2, 4, 8, 16, 31, 57, 99, 163, 256, 386, 562, ...
						\\ \cline{2-2}
						& $f(n) = \displaystyle\frac{(n^{4}-6n^{3}+23n^{2}-18{n}+24)}{24}$.
						\\ \hline

						\multirow{2}{2cm}{Secuencia de Narayana}
						& 1, 1, 1, 2, 3, 4, 6, 9, 13, 19, 28, 41, 60, 88, 129, ...
						\\ \cline{2-2}
						& $f(0) = f(1) = f(2) = 1; f(n) = f(n-1) + f(n-3)$ para todo $n>2$.
						\\ \hline

						\multirow{2}{2cm}{Suma de los divisores de un número}
						& 1, 3, 4, 7, 6, 12, 8, 15, 13, 18, 12, 28, 14, 24, ...
						\\ \cline{2-2}
						&Para todo $n>1$, 
						$n=\displaystyle p_{1}^{\textstyle a_{1}}\displaystyle p_{2}^{\textstyle a_{2}}...
						\displaystyle p_{k}^{\textstyle a_{k}}$ entonces:

						$f(n) = \displaystyle\frac{p_{1}^{a_{1} + 1} - 1}{p_{1} - 1} * \frac{p_{2}^{a_{2} + 1} - 1}{p_{2} - 1}
						* ... * \frac{p_{k}^{a_{k} + 1} - 1}{p_{k} - 1}$ 
						\\ \hline

						\multirow{2}{2cm}{Números de Super-Catalán} 
						& 1, 1, 3, 11, 45, 197, 903, 4279, 20793, 103049, 518859, ...
						\\ \cline{2-2} 
						& El número de formas de insertar paréntesis en una secuencia y el número de formas de partir un
						polígono convexo en polígonos más pequeños mediante la inserción de diagonales. f(1)=f(2)=1;

						$f(n) = \displaystyle\frac{3(2n-3)*f(n-1) - (n-3)*f(n-2)}{n}$
						\\ \hline

					\end{supertabular}
					}
				\end{center}
				
				\subsection{Secuencias}
						
					\textbf{Primos:}\\
					\vspace{3mm}
					2 3 5 7 11 13 17 19 23 29 31 37 41 43 47 53 59 61 67 71 73 79 83 89 97 101 103 107 109 113 127 131 137 139
					149 151 157 163 167 173 179 181 191 193 197 199 211 223 227 229 233 239 241 251 257 263 269 271 277 281 283
					293 307 311 313 317 331 337 347 349 353 359 367 373 379 383 389 397 401 409 419 421 431 433 439 443 449 457
					461 463 467 479 487 491 499 503 509 521 523 541 547 557 563 569 571 577 587 593 599 601 607 613 617 619 631
					641 643 647 653 659 661 673 677 683 691 701 709 719 727 733 739 743 751 757 761 769 773 787 797 809 811 821
					823 827 829 839 853 857 859 863 877 881 883 887 907 911 919 929 937 941 947 953 967 971 977 983 991 997
					1009 1013 1019 1021 1031 1033 1039 1049 1051 1061 1063 1069 1087 1091 1093 1097 1103 1109 1117 1123 1129
					1151 1153 1163 1171 1181 1187 1193 1201 1213 1217 1223 1229 1231 1237 1249 1259 1277 1279 1283 1289 1291
					1297 1301 1303 1307 1319 1321 1327 1361 1367 1373 1381 1399 1409 1423 1427 1429 1433 1439 1447 1451 1453
					1459 1471 1481 1483 1487 1489 1493 1499 1511 1523 1531 1543 1549 1553 1559 1567 1571 1579 1583 1597 1601
					1607 1609 1613 1619 1621 1627 1637 1657 1663 1667 1669 1693 1697 1699 1709 1721 1723 1733 1741 1747 1753 
					1759 1777 1783 1787 1789 1801 1811 1823 1831 1847 1861 1867 1871 1873 1877 1879 1889 1901 1907 1913 1931 
					1933 1949 1951 1973 1979 1987 1993 1997 1999 2003 2011 2017 2027 2029 2039 2053 2063 2069 2081 2083 2087 
					2089 2099 2111 2113 2129 2131 2137 2141 2143 2153 2161 2179 2203 2207 2213 2221 2237 2239 2243 2251 2267 
					2269 2273 2281 2287 2293 2297 2309 2311 2333 2339 2341 2347 2351 2357 2371 2377 2381 2383 2389 2393 2399 
					2411 2417 2423 2437 2441 2447 2459 2467 2473 2477

					\vspace{8mm}
					\textbf{Primos cercanos a potencias de 10:}\\
					\vspace{3mm}
					7 11, 89 97 101 103, 983 991 997 1009 1013 1019, 
					9941 9949 9967 9973 10007 10009 10037 10039 10061 10067 10069 10079, 
					99961 99971 99989 99991 100003 100019 100043 100049 100057 100069, 
					999959 999961 999979 999983 1000003 1000033 1000037 1000039,
					9999943 9999971 9999973 9999991 10000019 10000079 10000103 10000121,
					99999941 99999959 99999971 99999989 100000007 100000037 100000039 100000049,
					999999893 999999929 999999937 1000000007 1000000009 1000000021 1000000033
					
					\vspace{8mm}
					\textbf{Fibonacci:}\\
					\vspace{3mm}
					0 1 1 2 3 5 8 13 21 34 55 89 144 233 377 610 987 1597 2584 4181 6765 10946 17711 28657 46368 75025 121393 
					196418 317811 514229 832040 1346269 2178309 3524578 5702887 9227465 14930352 24157817 39088169 63245986 
					102334155 165580141 267914296 433494437 701408733 1134903170 1836311903
					
					\vspace{8mm}
					\textbf{Factoriales:}\\
					\vspace{3mm}
					1 2 6 24 120 720 5040 40320 362880 3628800 39916800 479001600 6227020800 87178291200 1307674368000 
					20922789888000 355687428096000 6402373705728000 121645100408832000
					
					\vspace{8mm}
					\textbf{Potencias de dos:} de 1 hasta 63\\
					\vspace{3mm}
					1 2 4 8 16 32 64 128 256 512 1024 2048 4096 8192 16384 32768 65536 131072 262144 524288 1048576 2097152 
					4194304 8388608 16777216 33554432 67108864 134217728 268435456 536870912 1073741824 2147483648 4294967296 
					8589934592 17179869184 34359738368 68719476736 137438953472 274877906944 549755813888 1099511627776 
					2199023255552 4398046511104 8796093022208 17592186044416 35184372088832 70368744177664 140737488355328 
					281474976710656 562949953421312 1125899906842624 2251799813685248 4503599627370496 9007199254740992 
					18014398509481984 36028797018963968 72057594037927936 144115188075855872 288230376151711744 
					576460752303423488 1152921504606846976 2305843009213693952 4611686018427387904 9223372036854775808

\end{document}


